% Options for packages loaded elsewhere
\PassOptionsToPackage{unicode}{hyperref}
\PassOptionsToPackage{hyphens}{url}
%
\documentclass[
]{article}
\usepackage{amsmath,amssymb}
\usepackage{lmodern}
\usepackage{iftex}
\ifPDFTeX
  \usepackage[T1]{fontenc}
  \usepackage[utf8]{inputenc}
  \usepackage{textcomp} % provide euro and other symbols
\else % if luatex or xetex
  \usepackage{unicode-math}
  \defaultfontfeatures{Scale=MatchLowercase}
  \defaultfontfeatures[\rmfamily]{Ligatures=TeX,Scale=1}
\fi
% Use upquote if available, for straight quotes in verbatim environments
\IfFileExists{upquote.sty}{\usepackage{upquote}}{}
\IfFileExists{microtype.sty}{% use microtype if available
  \usepackage[]{microtype}
  \UseMicrotypeSet[protrusion]{basicmath} % disable protrusion for tt fonts
}{}
\makeatletter
\@ifundefined{KOMAClassName}{% if non-KOMA class
  \IfFileExists{parskip.sty}{%
    \usepackage{parskip}
  }{% else
    \setlength{\parindent}{0pt}
    \setlength{\parskip}{6pt plus 2pt minus 1pt}}
}{% if KOMA class
  \KOMAoptions{parskip=half}}
\makeatother
\usepackage{xcolor}
\IfFileExists{xurl.sty}{\usepackage{xurl}}{} % add URL line breaks if available
\IfFileExists{bookmark.sty}{\usepackage{bookmark}}{\usepackage{hyperref}}
\hypersetup{
  pdftitle={Guide d'achat de diamants},
  hidelinks,
  pdfcreator={LaTeX via pandoc}}
\urlstyle{same} % disable monospaced font for URLs
\usepackage[margin=1in]{geometry}
\usepackage{color}
\usepackage{fancyvrb}
\newcommand{\VerbBar}{|}
\newcommand{\VERB}{\Verb[commandchars=\\\{\}]}
\DefineVerbatimEnvironment{Highlighting}{Verbatim}{commandchars=\\\{\}}
% Add ',fontsize=\small' for more characters per line
\usepackage{framed}
\definecolor{shadecolor}{RGB}{248,248,248}
\newenvironment{Shaded}{\begin{snugshade}}{\end{snugshade}}
\newcommand{\AlertTok}[1]{\textcolor[rgb]{0.94,0.16,0.16}{#1}}
\newcommand{\AnnotationTok}[1]{\textcolor[rgb]{0.56,0.35,0.01}{\textbf{\textit{#1}}}}
\newcommand{\AttributeTok}[1]{\textcolor[rgb]{0.77,0.63,0.00}{#1}}
\newcommand{\BaseNTok}[1]{\textcolor[rgb]{0.00,0.00,0.81}{#1}}
\newcommand{\BuiltInTok}[1]{#1}
\newcommand{\CharTok}[1]{\textcolor[rgb]{0.31,0.60,0.02}{#1}}
\newcommand{\CommentTok}[1]{\textcolor[rgb]{0.56,0.35,0.01}{\textit{#1}}}
\newcommand{\CommentVarTok}[1]{\textcolor[rgb]{0.56,0.35,0.01}{\textbf{\textit{#1}}}}
\newcommand{\ConstantTok}[1]{\textcolor[rgb]{0.00,0.00,0.00}{#1}}
\newcommand{\ControlFlowTok}[1]{\textcolor[rgb]{0.13,0.29,0.53}{\textbf{#1}}}
\newcommand{\DataTypeTok}[1]{\textcolor[rgb]{0.13,0.29,0.53}{#1}}
\newcommand{\DecValTok}[1]{\textcolor[rgb]{0.00,0.00,0.81}{#1}}
\newcommand{\DocumentationTok}[1]{\textcolor[rgb]{0.56,0.35,0.01}{\textbf{\textit{#1}}}}
\newcommand{\ErrorTok}[1]{\textcolor[rgb]{0.64,0.00,0.00}{\textbf{#1}}}
\newcommand{\ExtensionTok}[1]{#1}
\newcommand{\FloatTok}[1]{\textcolor[rgb]{0.00,0.00,0.81}{#1}}
\newcommand{\FunctionTok}[1]{\textcolor[rgb]{0.00,0.00,0.00}{#1}}
\newcommand{\ImportTok}[1]{#1}
\newcommand{\InformationTok}[1]{\textcolor[rgb]{0.56,0.35,0.01}{\textbf{\textit{#1}}}}
\newcommand{\KeywordTok}[1]{\textcolor[rgb]{0.13,0.29,0.53}{\textbf{#1}}}
\newcommand{\NormalTok}[1]{#1}
\newcommand{\OperatorTok}[1]{\textcolor[rgb]{0.81,0.36,0.00}{\textbf{#1}}}
\newcommand{\OtherTok}[1]{\textcolor[rgb]{0.56,0.35,0.01}{#1}}
\newcommand{\PreprocessorTok}[1]{\textcolor[rgb]{0.56,0.35,0.01}{\textit{#1}}}
\newcommand{\RegionMarkerTok}[1]{#1}
\newcommand{\SpecialCharTok}[1]{\textcolor[rgb]{0.00,0.00,0.00}{#1}}
\newcommand{\SpecialStringTok}[1]{\textcolor[rgb]{0.31,0.60,0.02}{#1}}
\newcommand{\StringTok}[1]{\textcolor[rgb]{0.31,0.60,0.02}{#1}}
\newcommand{\VariableTok}[1]{\textcolor[rgb]{0.00,0.00,0.00}{#1}}
\newcommand{\VerbatimStringTok}[1]{\textcolor[rgb]{0.31,0.60,0.02}{#1}}
\newcommand{\WarningTok}[1]{\textcolor[rgb]{0.56,0.35,0.01}{\textbf{\textit{#1}}}}
\usepackage{graphicx}
\makeatletter
\def\maxwidth{\ifdim\Gin@nat@width>\linewidth\linewidth\else\Gin@nat@width\fi}
\def\maxheight{\ifdim\Gin@nat@height>\textheight\textheight\else\Gin@nat@height\fi}
\makeatother
% Scale images if necessary, so that they will not overflow the page
% margins by default, and it is still possible to overwrite the defaults
% using explicit options in \includegraphics[width, height, ...]{}
\setkeys{Gin}{width=\maxwidth,height=\maxheight,keepaspectratio}
% Set default figure placement to htbp
\makeatletter
\def\fps@figure{htbp}
\makeatother
\setlength{\emergencystretch}{3em} % prevent overfull lines
\providecommand{\tightlist}{%
  \setlength{\itemsep}{0pt}\setlength{\parskip}{0pt}}
\setcounter{secnumdepth}{-\maxdimen} % remove section numbering
\ifLuaTeX
  \usepackage{selnolig}  % disable illegal ligatures
\fi

\title{Guide d'achat de diamants}
\author{}
\date{\vspace{-2.5em}}

\begin{document}
\maketitle

\textbf{Le fichier Rmd est à modifier !!!}

\emph{Vous DEVEZ ENLEVER les parties qui concernent les explications et
les exemples pour les REMPLACER par votre PROPRE TRAVAIL !!!!!!!}

N'oubliez pas d'enregistrer votre travail à la fin de la séance.

\hypertarget{description-rapide-du-jeu-de-donnuxe9es}{%
\subsection{Description rapide du jeu de
données}\label{description-rapide-du-jeu-de-donnuxe9es}}

Utiliser les informations du fichier donné sur Chamilo et/ou des
informations que vous pouvez trouver sur internet pour expliquer les
données dont vous disposez.

\hypertarget{description-statistique-des-diffuxe9rentes-variables}{%
\subsection{Description statistique des différentes
variables}\label{description-statistique-des-diffuxe9rentes-variables}}

En fonction du type de variable (qualitative/quantitative) vous
choisirez le graphique adapté ou les résumés numériques intéressants
(moyenne, médiane, déciles). Puis vous donnerez quelques commentaires.

\hypertarget{la-couleur}{%
\subsubsection{La couleur}\label{la-couleur}}

On peut faire apparaître le code R et le résultat de l'exécution ainsi :

\begin{Shaded}
\begin{Highlighting}[]
\CommentTok{\# combien de diamants pour chaque couleur ?}
\NormalTok{diamonds }\SpecialCharTok{\%\textgreater{}\%}
  \FunctionTok{group\_by}\NormalTok{(color) }\SpecialCharTok{\%\textgreater{}\%}
  \FunctionTok{summarise}\NormalTok{(}\AttributeTok{frequence =} \FunctionTok{round}\NormalTok{(}\FunctionTok{n}\NormalTok{() }\SpecialCharTok{/} \FunctionTok{nrow}\NormalTok{(diamonds), }\DecValTok{2}\NormalTok{), }\AttributeTok{nombre =} \FunctionTok{n}\NormalTok{())}
\end{Highlighting}
\end{Shaded}

\begin{verbatim}
## # A tibble: 7 x 3
##   color frequence nombre
##   <ord>     <dbl>  <int>
## 1 D          0.13   6775
## 2 E          0.18   9797
## 3 F          0.18   9542
## 4 G          0.21  11292
## 5 H          0.15   8304
## 6 I          0.1    5422
## 7 J          0.05   2808
\end{verbatim}

La couleur la moins fréquente est la J (la pire) avec 5\% des diamants.
Pour les autres couleurs la fréquence se situe entre 10\% (la couleur I)
et 21\% (la couleur G).

\hypertarget{la-qualituxe9-de-la-coupe}{%
\subsubsection{La qualité de la coupe}\label{la-qualituxe9-de-la-coupe}}

à compléter

\hypertarget{la-clartuxe9}{%
\subsubsection{La clarté}\label{la-clartuxe9}}

La clarté est une variable qualitative ordinale donc on peut faire un
diagramme en barres en ordonnant l'axe des x de la qualité la moins
bonne à la meilleure (ou l'inverse). Il peut aussi être intéressant de
regrouper des catégories de clarté comme ci-dessous :

\begin{Shaded}
\begin{Highlighting}[]
\CommentTok{\# Recoder la variable clarté}
\NormalTok{diamonds }\SpecialCharTok{\%\textgreater{}\%}
  \FunctionTok{mutate}\NormalTok{(}\AttributeTok{clarte =} \FunctionTok{recode}\NormalTok{(clarity,}
    \StringTok{"I1"} \OtherTok{=} \StringTok{"Visible"}\NormalTok{, }\StringTok{"SI1"} \OtherTok{=} \StringTok{"Petites inclusions"}\NormalTok{,}
    \StringTok{"SI2"} \OtherTok{=} \StringTok{"Petites inclusions"}\NormalTok{,}
    \StringTok{"VS1"} \OtherTok{=} \StringTok{"Minuscules inclusions"}\NormalTok{,}
    \StringTok{"VS2"} \OtherTok{=} \StringTok{"Minuscules inclusions"}\NormalTok{,}
    \StringTok{"VVS1"} \OtherTok{=} \StringTok{"Difficilement visibles"}\NormalTok{,}
    \StringTok{"VVS2"} \OtherTok{=} \StringTok{"Difficilement visibles"}\NormalTok{,}
    \StringTok{"IF"} \OtherTok{=} \StringTok{"Pur"}
\NormalTok{  )) }\SpecialCharTok{\%\textgreater{}\%}
  \FunctionTok{group\_by}\NormalTok{(clarte) }\SpecialCharTok{\%\textgreater{}\%}
  \FunctionTok{summarise}\NormalTok{(}\AttributeTok{frequence =} \FunctionTok{round}\NormalTok{(}\FunctionTok{n}\NormalTok{() }\SpecialCharTok{/} \FunctionTok{nrow}\NormalTok{(diamonds), }\DecValTok{2}\NormalTok{), }\AttributeTok{nombre =} \FunctionTok{n}\NormalTok{())}
\end{Highlighting}
\end{Shaded}

\begin{verbatim}
## # A tibble: 5 x 3
##   clarte                 frequence nombre
##   <ord>                      <dbl>  <int>
## 1 Visible                     0.01    741
## 2 Petites inclusions          0.41  22259
## 3 Minuscules inclusions       0.38  20429
## 4 Difficilement visibles      0.16   8721
## 5 Pur                         0.03   1790
\end{verbatim}

Commentaire ?

\hypertarget{le-nombre-de-carats}{%
\subsubsection{Le nombre de carats}\label{le-nombre-de-carats}}

\textbf{Mais qu'est-ce qu'un carat ??}

Exemples possibles de présentation : Les instructions R peuvent être
cachées avec la commande echo=FALSE, voici la moyenne et la valeur
médiane de la variable CARAT

\begin{verbatim}
## # A tibble: 1 x 7
##   minimum moyenne maximum    Q1 médiane    Q3    D9
##     <dbl>   <dbl>   <dbl> <dbl>   <dbl> <dbl> <dbl>
## 1     0.2   0.798    5.01   0.4     0.7  1.04  1.51
\end{verbatim}

Et un petit commentaire : on constate ici que la moyenne du nombre de
carats se situe à environ 8 et qu'elle est supérieure à la médiane
indiquant que certains diamants ont un nombre de carats grand ce que
l'on voit sur la valeur maximale qui est de 5 carats soit un diamant
d'environ 1 gramme.

\textbf{Que signifie que D9=1.51 ?}

On peut bien sûr inclure des graphiques :

\begin{Shaded}
\begin{Highlighting}[]
\FunctionTok{ggplot}\NormalTok{(diamonds) }\SpecialCharTok{+}
  \FunctionTok{geom\_histogram}\NormalTok{(}\FunctionTok{aes}\NormalTok{(}\AttributeTok{x =}\NormalTok{ carat, }\AttributeTok{y =}\NormalTok{ ..density..), }\AttributeTok{breaks =} \FunctionTok{seq}\NormalTok{(}\DecValTok{0}\NormalTok{, }\FloatTok{3.5}\NormalTok{, }\FloatTok{0.25}\NormalTok{), }\AttributeTok{fill =} \StringTok{"grey40"}\NormalTok{, }\AttributeTok{color =} \StringTok{"black"}\NormalTok{, }\AttributeTok{alpha =} \FloatTok{0.4}\NormalTok{)}
\end{Highlighting}
\end{Shaded}

\begin{verbatim}
## Warning: The dot-dot notation (`..density..`) was deprecated in ggplot2 3.4.0.
## i Please use `after_stat(density)` instead.
\end{verbatim}

\includegraphics{TP2_et_TP_3_2024_nom_prenom_files/figure-latex/exemples de graphiques-1.pdf}

\begin{Shaded}
\begin{Highlighting}[]
\FunctionTok{ggplot}\NormalTok{(diamonds, }\FunctionTok{aes}\NormalTok{(}\AttributeTok{x =}\NormalTok{ color, }\AttributeTok{y =}\NormalTok{ carat)) }\SpecialCharTok{+}
  \FunctionTok{geom\_boxplot}\NormalTok{()}
\end{Highlighting}
\end{Shaded}

\includegraphics{TP2_et_TP_3_2024_nom_prenom_files/figure-latex/exemples de graphiques-2.pdf}

En regardant de plus près, c'est-à-dire en précisant que chaque
intervalle sur l'axe des abscisses doit être de 0,01 et en ne
considérant que les diamants de moins de 1,51 carats (soit 90\% des
diamants), on constate ceci :

\begin{Shaded}
\begin{Highlighting}[]
\NormalTok{diamonds }\SpecialCharTok{\%\textgreater{}\%}
  \FunctionTok{filter}\NormalTok{(carat }\SpecialCharTok{\textless{}=} \FloatTok{1.51}\NormalTok{) }\SpecialCharTok{\%\textgreater{}\%}
  \FunctionTok{ggplot}\NormalTok{() }\SpecialCharTok{+}
  \FunctionTok{geom\_histogram}\NormalTok{(}\FunctionTok{aes}\NormalTok{(}\AttributeTok{x =}\NormalTok{ carat),}
    \AttributeTok{breaks =}\NormalTok{ (}\FunctionTok{seq}\NormalTok{(}\FloatTok{0.2}\NormalTok{, }\FloatTok{1.51}\NormalTok{, }\AttributeTok{by =} \FloatTok{0.01}\NormalTok{)),}
    \AttributeTok{closed =} \StringTok{"left"}\NormalTok{, }\AttributeTok{fill =} \StringTok{"grey40"}\NormalTok{, }\AttributeTok{color =} \StringTok{"black"}\NormalTok{, }\AttributeTok{alpha =} \FloatTok{0.4}
\NormalTok{  ) }\SpecialCharTok{+}
  \FunctionTok{scale\_x\_continuous}\NormalTok{(}\AttributeTok{breaks =} \FunctionTok{seq}\NormalTok{(}\FloatTok{0.2}\NormalTok{, }\FloatTok{1.51}\NormalTok{, }\AttributeTok{by =} \FloatTok{0.05}\NormalTok{)) }\SpecialCharTok{+}
  \FunctionTok{theme}\NormalTok{(}\AttributeTok{axis.text.x =} \FunctionTok{element\_text}\NormalTok{(}\AttributeTok{angle =} \DecValTok{45}\NormalTok{, }\AttributeTok{hjust =} \DecValTok{1}\NormalTok{))}
\end{Highlighting}
\end{Shaded}

\includegraphics{TP2_et_TP_3_2024_nom_prenom_files/figure-latex/unnamed-chunk-2-1.pdf}

\textbf{Commentaire ? Explications ?}

\hypertarget{le-pourcentage-de-profondeur}{%
\subsubsection{Le pourcentage de
profondeur}\label{le-pourcentage-de-profondeur}}

à compléter

\hypertarget{la-table}{%
\subsubsection{La table}\label{la-table}}

à compléter

\hypertarget{longueur-largeur-et-profondeur}{%
\subsubsection{Longueur, largeur et
profondeur}\label{longueur-largeur-et-profondeur}}

à compléter

\hypertarget{comment-sexplique-le-prix-dun-diamant---a-faire-en-suxe9ance-2}{%
\subsection{Comment s'explique le prix d'un diamant ? -\textgreater{} A
faire en séance
2}\label{comment-sexplique-le-prix-dun-diamant---a-faire-en-suxe9ance-2}}

Pourquoi prendre le log du prix ? On constate dans le tableau suivant
que l'étendue des prix est très grande.

\begin{Shaded}
\begin{Highlighting}[]
\NormalTok{diamonds}\SpecialCharTok{\%\textgreater{}\%}\FunctionTok{summarise}\NormalTok{(}\AttributeTok{minimum =} \FunctionTok{min}\NormalTok{(price),}\AttributeTok{Q1 =} \FunctionTok{quantile}\NormalTok{(price, }\FloatTok{0.25}\NormalTok{), }\AttributeTok{moyenne =} \FunctionTok{round}\NormalTok{(}\FunctionTok{mean}\NormalTok{(price), }\DecValTok{0}\NormalTok{),  }\AttributeTok{mediane =} \FunctionTok{median}\NormalTok{(price), }\AttributeTok{Q3 =} \FunctionTok{quantile}\NormalTok{(price, }\FloatTok{0.75}\NormalTok{), }\AttributeTok{maximum=}\FunctionTok{max}\NormalTok{(price))}
\end{Highlighting}
\end{Shaded}

\begin{verbatim}
## # A tibble: 1 x 6
##   minimum    Q1 moyenne mediane    Q3 maximum
##     <int> <dbl>   <dbl>   <dbl> <dbl>   <int>
## 1     326   950    3933    2401 5324.   18823
\end{verbatim}

Cela est aussi visible sur l'histogramme :

\begin{Shaded}
\begin{Highlighting}[]
\FunctionTok{ggplot}\NormalTok{(diamonds, }\FunctionTok{aes}\NormalTok{(}\AttributeTok{x =}\NormalTok{ price)) }\SpecialCharTok{+}
  \FunctionTok{geom\_histogram}\NormalTok{(}\AttributeTok{breaks =} \FunctionTok{seq}\NormalTok{(}\DecValTok{0}\NormalTok{, }\DecValTok{20000}\NormalTok{, }\DecValTok{1000}\NormalTok{), }\AttributeTok{fill =} \StringTok{"grey40"}\NormalTok{, }\AttributeTok{color =} \StringTok{"black"}\NormalTok{, }\AttributeTok{alpha =} \FloatTok{0.4}\NormalTok{)}
\end{Highlighting}
\end{Shaded}

\includegraphics{TP2_et_TP_3_2024_nom_prenom_files/figure-latex/histogramme du prix-1.pdf}

La distribution avec l'axe des x en échelle logarithmique :

\includegraphics{TP2_et_TP_3_2024_nom_prenom_files/figure-latex/histogramme avec echelle log-1.pdf}

Une autre façon de faire : tracer l'histogramme de log(price) mais là on
voit que l'axe des x n'est plus en dollars !!!

\includegraphics{TP2_et_TP_3_2024_nom_prenom_files/figure-latex/histogramme du log(prix)-1.pdf}

Quelques statistiques sur le logarithme du prix en fonction de la
qualité de la coupe :

\begin{verbatim}
## # A tibble: 5 x 3
##   cut       moyenne médiane
##   <ord>       <dbl>   <dbl>
## 1 Fair         8.09    8.10
## 2 Good         7.84    8.02
## 3 Very Good    7.80    7.88
## 4 Premium      7.95    8.07
## 5 Ideal        7.64    7.50
\end{verbatim}

Et quelques autres graphiques à titre d'exemples

\includegraphics{TP2_et_TP_3_2024_nom_prenom_files/figure-latex/graphiques-1.pdf}
\includegraphics{TP2_et_TP_3_2024_nom_prenom_files/figure-latex/graphiques-2.pdf}
\includegraphics{TP2_et_TP_3_2024_nom_prenom_files/figure-latex/graphiques-3.pdf}
\includegraphics{TP2_et_TP_3_2024_nom_prenom_files/figure-latex/graphiques-4.pdf}

\begin{verbatim}
## Warning: Removed 32 rows containing missing values (`geom_point()`).
\end{verbatim}

\includegraphics{TP2_et_TP_3_2024_nom_prenom_files/figure-latex/graphiques-5.pdf}

\end{document}
